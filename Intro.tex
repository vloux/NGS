%
%  Intro
%
%  Created by Valentin Loux on 2010-11-24.
%  Copyright (c) 2010 INRA. All rights reserved.
%
\documentclass{beamer}



\mode<presentation>
{
  \usetheme{Antibes}
  % or ...

  \setbeamercovered{transparent}
  % or whatever (possibly just delete it)
}

% Use utf-8 encoding for foreign characters
\usepackage[french]{babel} 
\usepackage[utf8]{inputenc}
\usepackage{times}
\usepackage[T1]{fontenc}
\usepackage{pgf}
\usepackage{pgfarrows,pgfnodes,pgfshade}

% Package for including code in the document
 \usepackage{listings}
 \usepackage{url}


%%%%%% Begin Document %%%%%%%%%%%%%

\title[NGS] % (optional, use only with long paper titles)
{Analyse de données issues de séquenceurs nouvelle génération}

%\subtitle
%{Une plateforme d'annotation de génomes bactériens}

%\author[V. Loux] % (optional, use only with lots of authors)
%{Valentin~Loux}

\institute[INRA-MIG] % (optional, but mostly needed)
{
  Unité Mathématique, Informatique et Génome\\
  INRA, Jouy en Josas
}

\date[25 Novembre 2010] % (optional, should be abbreviation of conference name)
{Formation NGS 2010}

 \pgfdeclareimage[height=0.5cm]{miglogo}{miglogo}
 \logo{\pgfuseimage{miglogo}}


% Delete this, if you do not want the table of contents to pop up at
% the beginning of each subsection:
\AtBeginSection[]
{
  \begin{frame}<beamer>
    \frametitle{Outline}
    \tableofcontents[currentsection]
  \end{frame}
}


% If you wish to uncover everything in a step-wise fashion, uncomment
% the following command: 

%\beamerdefaultoverlayspecification{<+->}


\begin{document}
	
	
\begin{frame}
  \titlepage
\end{frame}

\begin{frame}
  \frametitle{Programme}
  \tableofcontents
  % You might wish to add the option [pausesections]

\begin{itemize}
	\item 9h-9h30 Présentation des intervenants, Tour de table
	\item 9h30-10h30 Introduction aux NGS (Jean-François Gibrat)
	\item 10h30-11h Applications des NGS (Valentin Loux)
	\item 11h-11h15 Pause
	\item 11h15-12h30 TP (V.L., Julien Fayolle, Fabien Melchiore)
	\item 12h30-13h45 Repas
	\item 13h45-15h00 TP
	\item 15h-15h45 Mapping : algorithmes et choix d'outils (J. F.)
	\item 15h45-16h Pause
	\item 16h-18h Fin du TP et conclusions
\end{itemize}
\end{frame}


\begin{frame}
	\frametitle{Intervenants}
\begin{itemize}
	\item Jean-François Gibrat (jean-francois.gibrat@jouy.inra.fr)
	\item Valentin Loux (valentin.loux@jouy.inra.fr)
	\item Julien Fayolle (julien.fayolle@jouy.inra.fr)
	\item Fabien Melchiore (fabien.melchiore@jouy.inra.fr) 
\end{itemize}
\end{frame}
\end{document}
